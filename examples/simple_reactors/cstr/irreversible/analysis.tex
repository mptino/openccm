\documentclass[a4paper,12pt]{article}
\usepackage[margin=0.65in]{geometry}
\usepackage{amsmath,amssymb,amstext}
\usepackage[hidelinks]{hyperref}
\usepackage{graphicx}
\usepackage{float}

\usepackage{setspace}
\setlength{\parskip}{0.65cm}
\singlespacing

\setlength{\parindent}{0cm}

\begin{document}

\title{\texttt{OpenCCM} CSTR Validation: Irreversible Linear Reactions}
\author{Matthew Peres Tino}
\maketitle

\section{Reaction System}

This problem closely follows that of Example 8.3/8.4 from \emph{Elements of Chemical Reaction Engineering, 5th ed.} by H. Scott Fogler.
Suppose we have the following irreversible coupled linear reaction system:
\begin{center}
	$A\xrightarrow[]{k_1} B$

	$B\xrightarrow[]{k_2} C$
\end{center}
where $k_1 = 0.5$ $hr^{-1}$ and $k_2 = 0.2$ $hr^{-1}$. 
The reaction rates are easily derived, and are as follows:
\begin{align}
    r_A &= -k_1 C_A \\
    r_B &= k_1 C_A - k_2 C_B \\
    r_C &= k_2 C_B
\end{align}

\section{CSTR Mass Balance}

A transient mass balance for a species $P$ in a CSTR is as follows. 
\begin{align}
    \frac{dN_P}{dt} &= F_{in} - F_{out} + Vr_P\\
    V\frac{dC_P}{dt} &= Q P_{IN} - QC_P + Vr_P \\
    \frac{dC_P}{dt} &= \frac{P_{IN}}{\tau} - \frac{C_P}{\tau} + r_P
\end{align}
where $N_P$ is the number of moles, $F$ represents the molar flow rate in $mol/s$, $V$ is the volume of the CSTR, $r_P$ is the reaction rate, and $C_P$ and $P_{IN}$ represent the concentration of species $P$ and the inlet feed rate (which plays the role of a ``boundary condition''), respectively.
Solving for constant volume $V$ and volumetric feed rate $Q$ allows for simplifications as shown above, where $\tau = \frac{V}{Q}$.

\subsection{Mass Balance for $A$}
Using the formula above, the mass balance for species $A$ is: $$ \frac{dC_A}{dt} = \frac{A_{IN}}{\tau} - \frac{C_A}{\tau} -k_1 C_A$$
This is a first order ODE that is easily solved by the integrating factor method. 
Let $\alpha_A = \frac{A_{IN}}{\tau}$ and $\beta_A = \frac{k_1\tau + 1}{\tau} $.
\begin{align}
    \frac{dC_A}{dt} + \beta_A C_A &= \alpha_A \\
    \frac{d}{dt}(C_A e^{\beta_A t}) &= \alpha_A e^{\beta_A t} \\ 
    C_A(t) &= \frac{\alpha_A}{\beta_A} + \gamma e^{-\beta_A t}
\end{align}
Note both the initial condition ($C_{A0}$) and inlet feed rate ($A_{IN}$) = 2M. 
Applying the initial condition yields $\gamma = A_{IN} - \frac{\alpha_A}{\beta_A}$. 
After some simplifications, we obtain the analytical transient mass balance expression for $A$: 
\begin{equation}
    C_A(t) = \frac{A_{IN}}{1+ k_1\tau} (1+ k_1\tau e^{-\beta_A t})
\end{equation}

\subsection{Mass Balance for $B$}

For $B$, the initial concentration ($C_{B0}$) and inlet feed rate ($B_{IN}$) are both 0M. Therefore the mass balance is:
\begin{equation}
    \frac{dC_B}{dt} = \frac{-C_B}{\tau} + k_1C_A - k_2C_B
\end{equation}
We will now define $\beta_B = \frac{k_2\tau + 1}{\tau}$.
The mass balance for $B$ can now be written as:
\begin{equation}
    \frac{dC_B}{dt} = -\beta_B C_B + k_1\alpha_A + k_1^2 \tau \alpha_A e^{-\beta_A t}
\end{equation}
Again the integrating factor method is used to solve for the transient profile of $B$: 
\begin{align}
    \frac{dC_B}{dt} + \beta_B C_B &=  k_1\alpha_A + k_1^2 \tau \alpha_A e^{-\beta_A t}\\
    \frac{d}{dt}(C_B e^{\beta_B t}) &= k_1\alpha_A e^{\beta_B t} + k_1^2 \tau \alpha_A e^{(\beta_B-\beta_A) t}\\
    C_B(t) &= \frac{k_1\alpha_A}{\beta_B} + \frac{k_1^2 \tau \alpha_A}{\beta_B - \beta_A} e^{-\beta_A t} + \gamma e^{-\beta_B t} 
\end{align}
Implementing the initial condition $C_{B0} = 0M$ and solving for $\gamma$, the resulting mass balance is 
\begin{equation}
    C_B(t) = \frac{k_1 \alpha_A}{\beta_B}(1-e^{-\beta_B t}) + \frac{k_1^2 \tau \alpha_A}{\beta_B - \beta_A} (e^{-\beta_A t} - e^{-\beta_B t})
\end{equation}

\newpage 
\subsection{Mass Balance for $C$}

Species $C$ is similar to $B$ in that the initial concentration ($C_{C0}$) and inlet feed rate ($C_{IN}$) are both 0M. 
The mass balance is 
\begin{align}
    \frac{dC_c}{dt} &= \frac{-C_c}{\tau} + k_2 C_B\\
    \frac{dC_C}{dt} + \frac{C_c}{\tau} &= k_2 C_B
\end{align}
Again the integrating factor method is used to solve the ODE:
\begin{equation}
    C_c e^{t/\tau} = \frac{k_2 k_1 \alpha_A}{\beta_B} \int dt e^{t/\tau} - e^{t(1/\tau - \beta_B)} + 
        \frac{k_2 k_1^2 \tau \alpha_A}{\beta_B - \beta_A} \int dt e^{t (1/\tau - \beta_A)} - e^{t(1/\tau - \beta_B)} 
\end{equation}
After some trivial integration we have the mass balance:
\begin{equation}
    C_c(t) = \gamma e^{-t/\tau} + \frac{k_2 k_1 \alpha_A}{\beta_B}
    \left(\tau - \frac{e^{-\beta_B}t}{1/\tau - \beta_B}\right) \frac{k_2 k_1^2 \tau \alpha_A}{\beta_B - \beta_A} 
                \left(\frac{e^{-\beta_A t}}{1/\tau - \beta_A} - \frac{e^{-\beta_B t}}{1/\tau - \beta_B}\right)
\end{equation}
The constant of integration $\gamma$ is found through the initial condition $C_{C0} = 0$, yielding 
\begin{equation}
    \gamma = \frac{k_2 k_1 \alpha_A}{\beta_B}\left(\frac{1}{1/\tau - \beta_B} - \tau\right) + 
        \frac{k_2 k_1^2 \tau \alpha_A}{\beta_B - \beta_A}\left(\frac{1}{1/\tau - \beta_B} - \frac{1}{1/\tau - \beta_A}\right)
\end{equation}


\section{Simulation Setup}

Using the reaction system previously described, the following initial condtitions and inlet feed rates (``boundary conditions'') in units of [M] (concentration) employed for simulations are re-iterated here. 
Note that the final transient profiles for all species were derived using these values and not for generalized (variable) conditions.
\begin{align}
	a_{BC} &= a_{IC} = 2\\
	b_{BC} &= b_{IC} = 0\\
	c_{BC} &= c_{IC} = 0
\end{align}

The solution vectors $sol^a$ (analytical) and $sol^n$ (\texttt{OpenCCM}) use the same specific timesteps given by the numerical solver.
The error between these two solutions for a species computed at each timestep $i$ for $N$ total timesteps is a normalized absolute error:2
\begin{equation}
	err = \frac{\sum_i^N \lvert sol^a_i - sol^n_i \rvert}{N}
\end{equation}

\newpage
Simulations were run using varying tolerances of 1e-3, 1e-7, 1e-10, and 1e-13 for both absolute and relative tolerances.
Additionally, although stricter tolerances result in larger number of timesteps, this is accounted for in the error equation (normalization by $N$).

\begin{figure}[H]
    \includegraphics[width=0.49\linewidth]{tol_1e-03}\hfill
    \includegraphics[width=0.49\linewidth]{tol_1e-07}
\end{figure}
\begin{figure}[H]
    \includegraphics[width=0.49\linewidth]{tol_1e-10}\hfill
    \includegraphics[width=0.49\linewidth]{tol_1e-13}
\end{figure}

\section{Conclusions}

In conclusion:
\begin{itemize}
	\item \texttt{OpenCCM}'s CSTR implementation can handle irreversible linear reactions as validated by comparison to an analytically derived mass balance system.
	\item Increasing the relative \& absolute simulation tolerances significantly reduces the overall error between analytical and numerical solutions.
\end{itemize}

\end{document}
